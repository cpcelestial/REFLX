%   Filename    : chapter_1.tex 
\chapter{Introduction}
\label{sec:researchdesc}    %labels help you reference sections of your document

\section{Overview of the Current State of Technology}
\label{sec:overview}

\begin{doublespace}

In the realm of sports science, the enhancement of athletes' response time and agility remains a critical focus, as these attributes directly influence performance in dynamic, unpredictable environments such as team sports and combat disciplines \cite{Hassan:2023:ESF}. Traditional training methods, including cone drills and ladder exercises, have long been employed to improve these skills, yet they often fall short in replicating the rapid, stimulus-driven demands of real-game scenarios. Over the past decade, light-based reaction training systems—devices utilizing visual stimuli like LED lights to prompt immediate motor responses—have emerged as innovative tools to bridge this gap. Systems such as FITLIGHT, BlazePod, and XLiGHT have been critically analyzed for their design features, including sensor connectivity, battery life, and operational reliability, revealing strengths in portability and customization but limitations in diagnostic precision and validity \cite{Ezhov:2021:MLS}.

Empirical studies have demonstrated that these systems can significantly enhance visual-motor coordination, reaction speed, and cognitive functions. For instance, interventions using FITLIGHT in small-sided games have led to marked improvements in harmonic abilities (e.g., rhythmization and responsiveness) and basic skills like dribbling among young basketball players \cite{Hassan:2023:ESF}. Similarly, a 10-week FITLIGHT program improved reaction times and dribbling speeds in female basketball athletes, with effect sizes indicating substantial neural adaptations \cite{Hassan:2025:FTI}. In motorsport contexts, light-based reactive agility training has boosted selective attention, cognitive flexibility, and cardiorespiratory capacity in car racing drivers \cite{Horvath:2022:ARA}. A systematic review of visual training interventions, including light board and stroboscopic methods, further corroborates these benefits, reporting 5-27\% reductions in reaction time across various sports, with greater efficacy in elite and younger athletes \cite{Jothi:2025:EVT}. Reliability assessments of systems like BlazePod have also affirmed their validity for measuring simple and complex reactions in mixed martial arts (MMA) athletes, with high intraclass correlations supporting their use in training protocols \cite{Polechonski:2024:RVR}.

Despite these advancements, significant gaps persist in the literature. While light-based systems show promise in controlled settings, their predictive value for field-based reactive agility remains limited, as evidenced by weak correlations between laboratory reaction speeds and on-field performance in soccer players \cite{Broodryk:2025:LBR}. This suggests a disconnect between isolated visual stimuli and the multifaceted perceptual-cognitive demands of sports, highlighting the need for more integrated, sport-specific designs. Moreover, comparative analyses underscore inconsistencies in system performance, such as variable Bluetooth stability and sensor delays, which could undermine training reproducibility \cite{Ezhov:2021:MLS}. These issues are largely tied to the wireless protocols used by commercial systems—typically Bluetooth or Wi-Fi—both of which introduce transmission overhead and variable latency, limiting their suitability for real-time reaction measurement. This technological gap highlights the need to explore emerging low-latency communication protocols.

\end{doublespace}

\section{Problem Statement}
\label{sec:problemstatement}

\begin{doublespace}

While traditional methods such as cone and ladder drills have been widely used to enhance reaction time and agility, they often fail to simulate the complex, stimulus-driven conditions of real-game scenarios \cite{Hassan:2023:ESF}. Recent light-based reaction training systems have shown potential in improving visual-motor coordination and cognitive response \cite{Horvath:2022:ARA, Jothi:2025:EVT}. However, these systems remain limited by high costs, connectivity issues, and questionable transferability of laboratory-based improvements to on-field performance \cite{Broodryk:2025:LBR, Ezhov:2021:MLS}.

To address these limitations, this study bridges engineering innovation and applied sports performance research by incorporating ESP-NOW—a low-latency, connectionless wireless communication protocol developed by Espressif—in its development of a low-cost and customizable light-based reaction training system. By leveraging ESP-NOW’s novel capabilities, the system aims to eliminate the communication bottlenecks present in existing technologies and provide a more temporally accurate, cost-efficient, and reliable training tool. If such limitations in current systems persist, athletes and coaches will continue to rely on devices that cannot fully simulate realistic gameplay conditions nor guarantee measurement precision. Thus, the integration and evaluation of ESP-NOW become central to developing a responsive and scientifically validated reaction training system.

\end{doublespace}

\section{Research Objectives}
\label{sec:researchobjectives}

\subsection{General Objective}
\label{sec:generalobjective}

\begin{doublespace}

The goal of this study is to develop and evaluate a light-based reaction training system that enhances the response time and agility of athletes, specifically in racket sports such as badminton, tennis, and table tennis.

\end{doublespace}

\subsection{Specific Objectives}
\label{sec:specificobjectives}

\begin{doublespace}

Specifically, this study aims to:

\begin{enumerate}

   \item design and construct a device equipped with infrared sensors, RGB lights, and speakers while integrating ESP-NOW for accurate motion detection and response measurement,
	   
   \item develop a software application that manages the device operations, records performance data, and functions both online and offline,
   
   \item calibrate and test the system to ensure precision, responsiveness, and synchronization between hardware and software components.
   
   \item conduct experimental trials assessing the system’s effectiveness in improving athletes’ response time and agility compared to traditional training methods, and
   
   \item evaluate the system’s usability, functionality, and overall user satisfaction based on feedback from athletes and coaches.

\end{enumerate}

\end{doublespace}

\section{Scope and Limitations of the Research}
\label{sec:scopelimitations}

\begin{doublespace}

This study focuses on the design, development, and short-term evaluation of a programmable light-based reaction training system specifically tailored for racket sports, including badminton, tennis, and table tennis. These sports were selected because they demand rapid visual processing, anticipatory decision-making, and fine motor control—abilities strongly linked to reaction time and agility.

Experimental trials will be conducted in controlled indoor training environments, using drills that simulate racket-sport scenarios such as serve returns, directional changes, and split-step reactions. The prototype system will incorporate LED visual cues and integrated speakers to deliver multimodal stimuli, allowing assessment of both single and dual sensory response conditions.

The evaluation will be limited to short-term performance outcomes, measuring pre-post changes in reaction and movement response metrics following exposure to the prototype system. Long-term effects, such as learning retention, in-game transfer, or perceptual-cognitive adaptations, are beyond the scope of this research.

\end{doublespace}

\section{Significance of the Research}
\label{sec:significance}

\begin{doublespace}

This research integrates engineering, computer science, and sports science by developing a light- and sound-based reaction training system that enhances athletes’ response time and agility through multimodal stimuli and data-driven feedback. Through a technical lens, the study contributes to the computer science community by implementing real-time sensor processing, audio-visual cue synchronization, and a user-centered software design. The inclusion of both infrared sensors and speakers allows for dynamic and varied training scenarios that engage multiple sensory pathways, thereby improving cognitive-motor coordination. Compared to existing systems that rely solely on visual cues or require expensive proprietary hardware, this design offers a more versatile, customizable, and cost-efficient solution.

A central innovation of this study is the application of ESP-NOW as the primary wireless protocol for synchronizing training stations. Unlike traditional Bluetooth-based systems, ESP-NOW enables stable, connectionless, and ultra–low-latency communication, thereby ensuring accurate temporal measurements essential for reaction-time research. Its novel integration into a light-based athletic training device positions this study at the intersection of sports technology and modern IoT communication design.

From a societal perspective, the system democratizes access to advanced reaction training technologies by providing an affordable and portable tool suitable for athletes, coaches, and educational institutions. It can also serve as a supplementary device for rehabilitation programs that aim to improve motor control and sensory processing. By combining technical innovation with accessibility, this research promotes evidence-based athletic development and supports the broader integration of smart, adaptive technologies in sports and human performance training.

\end{doublespace}