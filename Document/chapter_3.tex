%   Filename    : chapter_3.tex 
\chapter{Research Methodology}

\begin{doublespace}

This study follows a step-by-step research process to develop and evaluate a light-based reaction training system for racket sports. The methodology includes gathering background information, consulting experts, and creating the initial system design. It then covers building the hardware prototype, developing the software, and testing the system with UPV athletes to measure reaction time and agility. Each stage is planned to ensure that the device is accurate, usable, and aligned with actual training needs. The process ends with analyzing the results and completing the final technical documentation.

\end{doublespace}

\section{Research Activities}

\begin{doublespace}

Research activities include inquiry, survey, research, brainstorming, canvassing, consultation, review, interview, observe, experiment, design, test, document, etc.

\end{doublespace}

\subsection{Research and Consultation}

\begin{doublespace}
	
	In this phase, the researchers will work closely with the UPV PE department as they will serve as consultants to ensure and validate the feasibility of the proposed system for racket sports. This stage will involve a comprehensive research on reaction time and agility, focusing on light- and sound-based systems. The consultations will take place in the UPV Covered Court, specifically in the PE faculty rooms. This stage is  important to establishing a strong theoretical foundation, identifying current technological gaps, and ensuring the design aligns with athletic performance needs.

\end{doublespace}

\subsection{Brainstorming and System Design}

\begin{doublespace}
	
	This phase will have the UPV Computer Science and PE Department as the resource persons. This stage will be about the development of the design concept integrating infrared sensors, LED light modules, and speakers for multimodal stimuli, as well as the UI of the software. The consultations and creation of ideas will take place at the researchers' personal workstations. This phase is to conceptualize an efficient, portable, and user-friendly device and software tailored for various racket sports scenarios.

\end{doublespace}

\subsection{Prototype Development and Programming}

\begin{doublespace}
	
	This phase will involve the researchers in working closely with the adviser to guide them in the development of the prototype. This stage will be about the assembly of the hardware components(IR sensors, microcontroller, LED indicators, and audio modules) and develop an accompanying application for control and data management. Algorithms will be designed to measure response latency and agility metrics accurately. This stage will take place at the researchers' personal workstations and the CAS computer laboratories. This stage aims to produce a working prototype capable of recording and analyzing athlete response data effectively, validating the proposed system’s technical feasibility.

\end{doublespace}

\subsection{Testing and Experimentation}

\begin{doublespace}
	
	In this phase, the researchers will work closely with the UPV PE department and atheletes as they will serve to be the main observers and participants for the trials. This is the stage where the researchers will conduct the controlled trials to measure response time and agility improvements among the participants using the device. Data will be collected across multiple sessions to ensure the reliability and repeatability. Statistical analysis will be used to evaluate system effectiveness. This will take place in the UPV covered court. This stage will give an empirical validation on the system's impact on athlete performance and refine device parameters for optimal training outcomes.
	

\end{doublespace}

\subsection{Evaluation and Refinement}

\begin{doublespace}
	
	The researchers will be working closely with technical advisors and sports performance analysts for guidance in evaluation. This is the stage where the researchers will give an analysis on the experimental results, identify potential error, and  gather feedback from the athletes and coaches. Redesigning and recalibration of system components based on user experience and performance data will also be included in this phase. This phase will take place at the researchers personal workstations and CAS computer laboratories. This stage will aid in the improvement of the system accuracy, usability, and durability, ensuring the device meets practical and scientific standards before final deployment.

\end{doublespace}

\subsection{Documentation and Reporting}

\begin{doublespace}
	
	This phase will involve the research adviser to help the researchers in proper documentation and reporting. In this stage, the researchers will compile all research findings, design specifications, performance data, and analysis into a formal technical report and academic paper. This will take place at the researchers' personal workstations. This phase will ensure transparency, replicabilty and academic dissemination of the study's result and methodologies. 

\end{doublespace}

\section{Calendar of Activities}

\begin{doublespace}

The Table \ref{tab:timetableactivities} presents the chronological schedule of research activities from January to July, outlining the key phases of the study—from research and system design to prototype development, testing, and documentation. Each activity is strategically planned to ensure systematic progress and timely completion of the light-based reaction training system project.

\end{doublespace}

%
%  the following commands will be used for filling up the bullets in the Gantt chart
%
\newcommand{\weekone}{\textbullet}
\newcommand{\weektwo}{\textbullet \textbullet}
\newcommand{\weekthree}{\textbullet \textbullet \textbullet}
\newcommand{\weekfour}{\textbullet \textbullet \textbullet \textbullet}

%
%  alternative to bullet is a star 
%
\begin{comment}
   \newcommand{\weekone}{$\star$}
   \newcommand{\weektwo}{$\star \star$}
   \newcommand{\weekthree}{$\star \star \star$}
   \newcommand{\weekfour}{$\star \star \star \star$ }
\end{comment}



\begin{table}[ht]   %t means place on top, replace with b if you want to place at the bottom
\centering
\caption{Timetable of Activities} \vspace{0.25em}
\begin{tabular}{|p{2in}|c|c|c|c|c|c|c|c|} \hline
\centering Activities (2026) & Jan   & Feb & Mar & Apr & May & Jun & Jul \\ \hline
Research and Consultation      & ~~~\weektwo &  &  &  &  &  &  \\ \hline
Brainstorming and System Design & ~~~\weektwo  & \weekone~~~ &  &  &  &  &  \\ \hline
Prototype Development and Programming      &  & \weekthree &  &  &  &  &  \\ \hline
Testing and Experimentation     &  & ~~~\weekone & \weekfour &  &  &  &  \\ \hline
Evaluation and Refinement      &   &  &  & \weekfour & \weekfour &  &  \\ \hline
Documentation and Reporting & ~~~\weektwo  & \weekfour & \weekfour & \weekfour & \weekfour & \weekfour & \weektwo~~~ \\ \hline
\end{tabular}
\label{tab:timetableactivities}
\end{table}

