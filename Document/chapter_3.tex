%   Filename    : chapter_3.tex 
\chapter{Research Methodology}
\label{sec:methodology}

\begin{doublespace}

This study follows a step-by-step research process to develop and evaluate a light-based reaction training system for racket sports. The methodology includes gathering background information, consulting experts, and creating the initial system design. It then covers building the hardware prototype, developing the software, and testing the system with UPV athletes to measure reaction time and agility. Each stage is planned to ensure that the device is accurate, usable, and aligned with actual training needs. The process ends with analyzing the results and completing the final technical documentation.

\end{doublespace}

\section{Research Activities}

\begin{doublespace}

Research activities include inquiry, survey, research, brainstorming, canvassing, consultation, review, interview, observe, experiment, design, test, document, etc.

\end{doublespace}

\subsection{Research and Consultation}

\begin{doublespace}
	
	The researchers will work closely with the UPV PE department as they will serve as consultants to ensure and validate the feasibility of the proposed system for racket sports. This stage will involve a comprehensive research on reaction time training that utilizes light-based systems, focusing those for racket sports. The consultations will take place in the UPV Covered Court, specifically in the PE faculty rooms. This stage is  important to establishing a strong theoretical foundation, identifying gaps, and ensuring the design aligns with athletic performance needs.

\end{doublespace}

\subsection{Brainstorming and System Design}

\begin{doublespace}
	
	This phase will involve resource persons from the UPV Computer Science and PE Departments in developing the final design concept of the system, including the integration of infrared sensors, LED light modules, and speakers. The user interface of the system software will be done in this phase as well. Consultations and brainstorming sessions will be carried out online or face-to-face with the goal being to conceptualize an efficient, portable, and user-friendly hardware–software system tailored to the specific training needs of racket sports.

\end{doublespace}

\subsection{Latency Benchmarking}

\begin{doublespace}
	
	This phase will involve resource persons from the UPV Computer Science and PE Departments in developing the final design concept of the system, including the integration of infrared sensors, LED light modules, and speakers. The user interface of the system software will be done in this phase as well. Consultations and brainstorming sessions will be carried out online or face-to-face with the goal being to conceptualize an efficient, portable, and user-friendly hardware–software system tailored to the specific training needs of racket sports.
	
\end{doublespace}

\subsection{System Development and Programming}

\begin{doublespace}
	
	The researchers will be working closely with their adviser to guide the development of the system. It involves assembling the hardware components and developing the accompanying application responsible for system control and data management. If benchmarking results prove to be positive, this phase will also involve integrating ESP-NOW as the wireless communication protocol of the system. The goal of this phase is to produce functional devices capable of reliably recording and analyzing athlete response data, thereby demonstrating the technical feasibility of the proposed system.

\end{doublespace}

\subsection{Testing and Experimentation}

\begin{doublespace}
	
	This phase is where the researchers will work together with the UPV PE Department and the participating athletes, who will serve as the primary observers and respondents during the trials. Controlled testing sessions will be conducted to measure reaction time and agility improvements using the developed device. Data will be gathered across multiple sessions to ensure reliability and repeatability, and statistical analyses will be performed to evaluate the system’s effectiveness. All testing activities will take place at the UPV Covered Court. This phase provides empirical validation of the system’s impact on athlete performance and helps identify necessary adjustments to optimize the device for training use.
	
\end{doublespace}

\subsection{Evaluation and Refinement}

\begin{doublespace}
	
	In this phase, the researchers will work closely with their adviser and the UPV PE Department to guide the evaluation process. The researchers will analyze the experimental results, identify potential sources of error, and gather feedback from athletes and coaches. This stage also involves redesigning and recalibrating system components based on user experience and performance data. The work will be conducted at the researchers’ personal workstations and in the CAS computer laboratories. This phase aims to enhance the system’s accuracy, usability, and durability, ensuring that the device meets both practical and scientific standards before final deployment.

\end{doublespace}

\subsection{Documentation and Reporting}

\begin{doublespace}
	
	This phase will involve the research adviser to ensure that proper documentation and reporting will be done by the researchers. They will compile all research findings, design specifications, performance data, and analyses into a formal technical report and academic paper. This work will be completed at the researchers’ personal workstations. The purpose of this phase is to ensure transparency, replicability, and effective academic dissemination of the study’s results and methodologies.
	
\end{doublespace}

\section{Calendar of Activities}

\begin{doublespace}

The Table \ref{tab:timetableactivities} presents the chronological schedule of research activities from January to July, outlining the key phases of the study—from research and system design to prototype development, testing, and documentation. Each activity is strategically planned to ensure systematic progress and timely completion of the light-based reaction training system project.

\end{doublespace}

%
%  the following commands will be used for filling up the bullets in the Gantt chart
%
\newcommand{\weekone}{\textbullet}
\newcommand{\weektwo}{\textbullet \textbullet}
\newcommand{\weekthree}{\textbullet \textbullet \textbullet}
\newcommand{\weekfour}{\textbullet \textbullet \textbullet \textbullet}

%
%  alternative to bullet is a star 
%
\begin{comment}
   \newcommand{\weekone}{$\star$}
   \newcommand{\weektwo}{$\star \star$}
   \newcommand{\weekthree}{$\star \star \star$}
   \newcommand{\weekfour}{$\star \star \star \star$ }
\end{comment}



\begin{table}[ht]   %t means place on top, replace with b if you want to place at the bottom
\centering
\caption{Timetable of Activities} \vspace{0.25em}
\begin{tabular}{|p{2in}|c|c|c|c|c|c|c|c|} \hline
\centering Activities (2026) & Jan   & Feb & Mar & Apr & May & Jun & Jul \\ \hline
Research and Consultation      & ~~~\weektwo &  &  &  &  &  &  \\ \hline
Brainstorming and System Design & ~~~\weektwo  & \weekone~~~ &  &  &  &  &  \\ \hline
Prototype Development and Programming      &  & \weekthree &  &  &  &  &  \\ \hline
Testing and Experimentation     &  & ~~~\weekone & \weekfour &  &  &  &  \\ \hline
Evaluation and Refinement      &   &  &  & \weekfour & \weekfour &  &  \\ \hline
Documentation and Reporting & ~~~\weektwo  & \weekfour & \weekfour & \weekfour & \weekfour & \weekfour & \weektwo~~~ \\ \hline
\end{tabular}
\label{tab:timetableactivities}
\end{table}

