%   Filename    : abstract.tex 
\begin{abstract}

\begin{doublespace}

Traditional training methods, such as cone and ladder drills, inadequately simulate the rapid, stimulus-driven demands of real-game scenarios in racket sports like badminton, tennis, and table tennis. While commercial light-based systems like FITLIGHT and BlazePod show promise in enhancing visual-motor coordination, reaction speed, and agility, they are hindered by high costs, connectivity issues, and limited transferability from lab to field performance. This research proposes an integrated hardware–software solution utilizing infrared sensors, RGB LED modules, and audio cues to deliver multimodal stimuli capable of simulating sport-specific decision-making demands. The study outlines the design, construction, and calibration of a wireless IoT-enabled prototype employing ESP-NOW communication for real-time responsiveness and synchronized data capture.

Experimental trials will evaluate the system’s effectiveness in improving reaction time and agility compared with traditional methods, while usability assessments will gather insights from athletes and coaches to refine functionality and user experience. By merging principles from computer science, sports science, and human–computer interaction, this study contributes to the development of accessible, evidence-based athletic training technologies and provides a foundation for future enhancements in adaptive, data-driven performance systems.

\end{doublespace}

\begin{flushleft}
\begin{tabular}{lp{4.25in}}
\hspace{-0.5em}\textbf{Keywords:}\hspace{0.25em} & Wireless integrated network sensors, sensor networks, applied computing, human-centered computing, interactive systems and tools \\
\end{tabular}
\end{flushleft}
\end{abstract}
